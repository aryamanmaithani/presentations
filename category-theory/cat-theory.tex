\documentclass[11pt,leqno,landscape,semhelv]{seminar}
\usepackage{amsmath, amssymb, amsfonts, amsthm, mathtools}
\usepackage{fancybox}
\usepackage[inline]{enumitem}
\usepackage{cancel}
\usepackage{soul}
\usepackage{centernot}
\usepackage{tikz-cd}
\usepackage{datetime2}

\theoremstyle{definition}
\newtheorem{joke}{Joke}
\newtheorem{thm}{Theorem}
\newtheorem{lem}[thm]{Lemma}
\newtheorem{cor}[thm]{Corollary}
\newtheorem{prop}[thm]{Proposition}
\newtheorem{defn}[thm]{Definition}

\usepackage{chngcntr}
\numberwithin{joke}{section}
\numberwithin{thm}{section}
\numberwithin{equation}{section}

\newcommand{\example}[1]{\refstepcounter{thm}\par\medskip
   {\textbf{Example \thethm.} #1} \rmfamily}
\newcommand{\remark}[1]{\refstepcounter{thm}\par\medskip
   {\textit{Remark \thethm.} #1} \rmfamily}
\usepackage{titlesec}
\titleformat{\section}[block]
  {}{\centering\huge\S\thesection}{0.25cm}{\centering\Huge\textsc}
\titleformat{\subsection}[block]
  {}{\S\S\thesubsection}{0.25cm}{\Large}
  
\renewcommand{\sec}[1]{%
\begin{slide}
\begin{center}
    \begin{center}
        \section{#1}
    \end{center}
\end{center}
\end{slide}}
\newcommand{\cd}[1]{
\begin{center}
	\begin{tikzcd}
		#1
	\end{tikzcd}
\end{center}}
\newcommand{\cod}{\operatorname{cod}}
\newcommand{\dom}{\operatorname{dom}}
\newcommand{\id}{\operatorname{id}}
\newcommand{\Hom}{\operatorname{Hom}}
\newcommand{\op}{^{\operatorname{op}}}
\newcommand{\mono}{\rightarrowtail}
\newcommand{\epi}{\twoheadrightarrow}
\let\emptyset\varnothing

\DeclareSymbolFont{matha}{OML}{txmi}{m}{it}
\DeclareMathSymbol{\varv}{\mathord}{matha}{35}

\setlength\parindent{0pt}

\usepackage{xcolor}
\definecolor{mybgcolor}{RGB}{50, 50, 50} %46, 51, 63
\definecolor{mylinkcolor}{RGB}{0, 255, 255} %46, 51, 63

\usepackage{pagecolor}
%\pagecolor{mybgcolor}
%\color{white}

\usepackage[colorlinks=true,
	%linkcolor=mylinkcolor
	]
	{hyperref}


\title{\vspace{1cm} Category Theory}
\author{Aryaman Maithani\\\url{https://aryamanmaithani.github.io/}}
\date{\DTMnow}

\begin{document}
\maketitle
\newpage
\setcounter{section}{-2}
\tableofcontents
\sec{Introduction}
This is essentially a way to force myself to productive and self-study Category Theory. I try to write these notes in a way that I'm teaching them to someone to help me understand it better.\\
That being said, I don't really think that these notes would be better than reading the book which I'm reading itself. The book being - 
\begin{center}
	\emph{Category Theory} by \emph{Steve Awodey}.
\end{center}
I do skip some of the more advanced examples but in places I do add extra explanations as I feel necessary. So yeah, have fun.
\sec{Preliminaries}
These are not strictly preliminaries, in the sense that you do not \emph{need} to know these to understand the text. However, since the examples do rely on these, it's certainly beneficial to know these.\\
The reader may skip these at the beginning and return to these whenever the text demands it.
\subsection{Axiom of Choice}
Given a collection $\mathcal{B}$ of nonempty sets, there exists a function
\begin{equation*} 
	f:\mathcal{B} \to \bigcup_{B \in \mathcal{B}}B
\end{equation*}
such that $f(B) \in B,$ for each $B \in \mathcal{B}.$\\
The function $f$ is called a \emph{\textbf{choice function}} for the collection $\mathcal{B}.$\\
Note that the above can be stated in an equivalent way where the collection of (nonempty) sets is being indexed by some set. This is done in the following manner:\\
Let $X$ be a set and $(E_x)_{x \in X}$ be an arbitrary collection of nonempty sets. \\
Set $E = \displaystyle\bigcup_{x \in X}E_x.$ \\
Then, there exists a function
\begin{equation*} 
 	f : X \to E
\end{equation*}
such that $f(x) \in E_x,$ for all $x \in X.$\\
As before, $f$ is called a \emph{\textbf{choice function}}.\\~\\
Informally, the definition above states that:\\
Given a collection $\mathcal{B}$ of nonempty sets, there is a way to ``choose'' exactly one element from each set. (Note that we do not demand the sets to be disjoint, so it is possible that the same element can picked from two different sets. That is okay.) \\
One may also note that there is no uniqueness demanded of the choice function.
\subsection{Monoids}
A \emph{monoid} is a set $M$ equipped with a binary operation $*$ and a distinguished element $u \in M$ satisfying:
\begin{itemize}
	\item $a*(b*c) = (a*b)*c,$ for all $a, b, c \in M,$ and
	\item $a*u = a = u*a$ for all $a \in M.$
\end{itemize}
We may sometimes also refer to the monoid as $(M, *, u).$\\
A monoid homomorphism $h:(M, *, u_M) \to (N, \cdot, u_N)$ is a function $h:M\to N$ satisfying
\begin{itemize}
	\item $h(a*b) = h(a)\cdot h(b),$ for all $a, b \in M,$ and
	\item $h(u_M) = h(u_N).$
\end{itemize}
\subsection{Groups}
With the same notation as earlier, a monoid $M$ is said to be \emph{group} if for every $a \in M,$ there exists some $b \in M$ such that $a * b = u = b * a.$\\
A group homomorphism is a monoid homomorphism between groups.
\subsection{Preorders}
A \emph{preorder} is a set $P$ equipped with a binary relation $\le$ satisfying:
\begin{itemize}
	\item $a \le a$ for all $a \in P,$ and
	\item $a \le b$ and $b \le c$ implies $a \le c$ for all $a, b, c \in P.$
\end{itemize}
That is, the relation is reflexive and transitive.
\subsection{Posets}
A \emph{poset} is a preorder $P$ with the additional condition that $\le$ is antisymmetric, that is,
	\begin{equation*} 
		a \le b \text{ and } b \le a \implies a = b \text{ for all } a, b \in P.
	\end{equation*}
An order-preserving map between posets $(P, \le_P)$ and $(Q, \le_Q)$ is a function $f:P\to Q$ satisfying
\begin{align*} 
	p \le_P q \implies f(p) \le_Q f(q). 
\end{align*}
\subsection{Boolean algebra} \label{boolalg}
A \emph{Boolean algebra} is a poset $B$ equipped with distinguished elements $0, 1,$ binary operations $a\vee b$ of ``join'' and $a\wedge b$ of ``meet,'' and a unary operation $\neg b$ of ``complementation.'' These are required to satisfy the conditions
\begin{align*} 
	0 &\le a\\
	a &\le 1\\
	a \le c \text{ and } b \le c \quad &\text{iff} \quad a \vee b \le c \\
	c \le a \text{ and } c \le b \quad &\text{iff} \quad c \le a \wedge b \\
	a \le \neg b \quad &\text{iff} \quad a \wedge b = 0\\
	\neg\neg a  &= a.
\end{align*}
A typical example of a Boolean algebra is the power-set $\mathcal{P}(X)$ of $X.$ We have the following identifications:
\begin{equation*} 
	0 = \emptyset, 1 = X, A \vee B = A \cup B, A \wedge B = A \cap B, \neg A = X\setminus A.
\end{equation*}
A Boolean homomorphism is a function $h:B\to B'$ between Boolean algebras that is an order-preserving map which preserves the addition structure, id est, 
\begin{align*} 
	h(0) &= 0, \\
	h(1) &= 1, \\
	h(a \vee b) &= h(a) \vee h(b), \\
	h(a \wedge b) &= h(a) \wedge h(b), \text{ and} \\
	h(\neg a) &= \neg h(a).
\end{align*}
%
\subsection{Topological spaces}
A \textbf{\emph{topology}} on a set $X$ is a collection $\mathcal{T}$ of subsets of $X$ having the following properties:
\begin{itemize} 
\item $\emptyset$ and $X$ are in $\mathcal{T}.$
\item The union of elements of any subcollection of $\mathcal{T}$ is in $\mathcal{T}.$
\item The intersection of the elements of any finite subcollection of $\mathcal{T}$ is in $\mathcal{T}.$
\end{itemize}
If $X$ is a topological space with topology $\mathcal{T}$, we say that a subset $U$ of $X$ is an \textbf{\emph{open set}} of $X$ if $U$ belongs to the collection $\mathcal{T}.$\\
In this text, we shall write $O(X)$ to denote $\mathcal{T},$ id est, the collection of open sets.
%
\subsection{Equivalence relations and Quotients} \label{equivrel}
Let $X$ be a set. A \emph{relation} $R$ \emph{on} $X$ is a subset $R \subset X \subset X.$ We write ``$aRb$'' to mean ``$(a, b) \in R$.''\\
Let $\sim$ be a relation on $X.$ We say that $\sim$ is an \emph{equivalence relation} on $X$ if the following holds for all $x, y, z \in X:$
\begin{itemize}
	\item (Reflexive) $x \sim x,$
	\item (Symmetric) $x \sim y$ implies $y \sim x,$
	\item (Transitive) $x \sim y$ and $y \sim z$ implies $x \sim z.$
\end{itemize}
Given such a relation, one defines the \emph{equivalence class} $[x]$ of an element $x \in X$ by

\begin{equation*} 
	[x] = \{y \in  X \mid x \sim y\}.
\end{equation*}

The various different equivalence classes $[x]$ then form a \emph{partition} of $X,$ that is:
\begin{itemize}
	\item Each equivalence class is nonempty,
	\item Any two distinct equivalence classes are disjoint, and
	\item The union of all equivalence classes is $X.$
\end{itemize}
(That the above is true is an exercise.)\\
In particular, given any $x \in X,$ there exists a unique equivalence class to which $x$ belongs.\\
The set of all equivalence classes

\begin{equation*} 
	X/\sim = \{[x] \mid x \in X\}
\end{equation*}
is called the \emph{quotient} of $X$ by $\sim.$ it is used in place of $X$ when one wants to ``abstract away'' the difference between equivalent elements $x \sim y,$ in the sense that in $X/\sim$ such (and only such) elements are identified, since

\begin{equation*} 
	[x] = [y] \; \text{iff} \; x \sim y.
\end{equation*}
Observe that the \emph{quotient mapping},

\begin{equation*} 
	q:X \to X/\sim
\end{equation*}
defined as $x \mapsto [x]$ has the property that a map $f:X\to Y$ extends along w,

\begin{equation*} 
	\begin{tikzcd}
		X \arrow[rr, "q"] \arrow[rrddd, "f"'] && X/\sim \arrow[ddd, dotted]\\
		&&\\&&\\
		&& Y
	\end{tikzcd}
\end{equation*}
precisely when $f$ respects the equivalence relation, in the sense that $x \sim y$ implies $f(x) = f(y).$ (The converse need not be true.)

This concept extends to many other cases of taking more sophisticated quotients in the cases of monoids, groups, topological spaces, et cetera.
\input{sec1.tex}
\input{sec2.tex}
\sec{Duality}
\subsection{The Duality Principle}
Let us recall the definition of a category: There are two kinds of \emph{things}, objects $A, B, C,\ldots$ and arrows $f, g, h, \ldots;$ four operations $\dom(f), \cod(f), 1_A, g \circ f;$ and these satisfy the following seven axioms:
\begin{align} 
  \dom(1_A) = A &\qquad \cod(1_A) = A\nonumber\\
  f\circ1_{\dom(f)} = f &\qquad 1_{\cod(f)}\circ f = f \label{catax}\\
  \dom(g\circ f) = \dom(f) &\qquad \cod(g\circ f) = \cod(g)\nonumber\\
  h\circ(g\circ f) &= (h\circ g) \circ f\nonumber
\end{align}
Where the operation ``$g\circ f$'' is defined precisely when
\begin{equation*} 
  \dom(g) = \cod(f),
\end{equation*}
so a suitable form of this should occur as a condition on each equation containing $\circ,$ as in $\dom(g) = \cod(f) \implies \dom(g\circ f) = \dom(f).$\\
Now, given any sentence $\Sigma$ in the elementary language of category theory, we can form the ``dual statement'' $\Sigma^*$ by making the following replacements:
\begin{align*} 
  f\circ g \; &\text{for} \; g \circ f,\\
  \cod \; &\text{for} \; \dom,\\
  \dom \; &\text{for} \; \cod.
\end{align*}
It is easy to see that after these replacements, the statement will again be well formed. Next, suppose that we have shown a statement $\Sigma$ to entail one $\Delta,$ that is, $\Sigma \implies \Delta,$ without using any of the category axioms. Then, it follows that $\Sigma^* \implies \Delta^*.$ This is because the substituted terms are mere undefined constants if we don't use any category axioms.\\
However, now observe that the axioms (\ref{catax}) for category theory (CT) are themselves ``self-dual,'' in the sense that we have,
\begin{equation*} 
  \text{CT}^* = \text{CT}.
\end{equation*}
We now have the following \emph{duality principle}.
%
\begin{prop}[formal duality]
  For any sentence $\Sigma$ in the language of category theory, if $\Sigma$ follows from the axioms of categories, then do foes its dual $\Sigma^*$:
  \begin{equation*} 
    \text{CT} \implies \Sigma \; \text{implies} \; \text{CT} \implies \Sigma^*.
  \end{equation*}
\end{prop}
Taking a more conceptual point of view, note that if a statement $\Sigma$ involves some diagram of objects and arrows,
\begin{equation*} 
  \begin{tikzcd}
    A \arrow[rr, "f"] \arrow[rrddd, "g\circ f"'] && B\arrow[ddd, "g"]\\
    &&\\&&\\
    &&C
  \end{tikzcd}
\end{equation*}
then the dual statement $\Sigma^*$ involves the diagram obtained from it by reversing the direction and the order of composition of arrows.
\begin{equation*} 
  \begin{tikzcd}
  A &  & B \arrow[ll, "f"']                            \\
    &  &                                               \\
    &  &                                               \\
    &  & C \arrow[lluuu, "f\circ g"] \arrow[uuu, "g"']
  \end{tikzcd}
\end{equation*}
Recalling the opposite category $\mathbf{C}\op$ of a category $\mathbf{C},$ we see that an interpretation of a statement $\Sigma$ in $\mathbf{C}$ automatically gives an interpretation of $\Sigma^*$ in $\mathbf{C}\op.$\\
Now suppose that a statement $\Sigma$ holds for all categories $\mathbf{C}.$ Then, it also holds in all categories $\mathbf{C}\op,$ and so $\Sigma^*$ holds in all categories $(\mathbf{C}\op)\op.$ But since for every category $\mathbf{C},$
\begin{equation*} 
  (\mathbf{C}\op)\op= \mathbf{C},
\end{equation*}
we see that $\Sigma^*$ also holds in all categories $\mathbf{C}.$ We therefore have the following form of conceptual form of the duality principle.
\begin{prop}[conceptual duality]
  For any statement $\Sigma$ about categories, if $\Sigma$ holds for all categories, then so does the dual statement $\Sigma^*.$
\end{prop}
%
\subsection{Coproducts}
Let us consider the example of products and see what the dual notion must be. We first recall the definition of product.
\begin{defn} 
  A diagram \begin{tikzcd}A & P \arrow[l, "p_1"'] \arrow[r, "p_2"] & B\end{tikzcd} is a \emph{product} of $A$ and $B,$ if for any $Z$ and \begin{tikzcd}A & Z \arrow[l, "z_1"'] \arrow[r, "z_2"] & B\end{tikzcd} there is a unique $u:Z\to P$ with $p_i \circ u = z_i,$ all as indicated in
  \begin{equation*} 
    \begin{tikzcd}
    &  & Z \arrow[llddd, "z_1"'] \arrow[rrddd, "z_2"] \arrow[ddd, "u", dotted] &  &   \\
    &  &                                                                       &  &   \\
    &  &                                                                       &  &   \\
  A &  & P \arrow[ll, "p_1"] \arrow[rr, "p_2"']                                &  & B
  \end{tikzcd}
  \end{equation*}
\end{defn}
Now what is the dual statement? The reader is encouraged to write the dual statement themselves and compare it with the next definition. The convention is to use the prefix ``co-'' to indicate the dual notion. Thus, we get the definition of a \emph{co-}product as follows.
\begin{defn} 
A diagram \begin{tikzcd}A \arrow[r, "q_1"] & Q & B \arrow[l, "q_2"']\end{tikzcd} is a \emph{coproduct} of $A$ and $B,$ if for any $Z$ and \begin{tikzcd}A \arrow[r, "z_1"] & Z & B \arrow[l, "z_2"']\end{tikzcd} there is a unique $u:Q\to Z$ with $u \circ q_i = z_i,$ all as indicated in

  \begin{equation*} 
    \begin{tikzcd}
    && Z&&\\
    &&  &&\\&&  &&\\
    A \arrow[rr, "q_1"'] \arrow[rruuu, "z_1"] &  & Q \arrow[uuu, "u"', dotted] &
    & B \arrow[ll, "q_2"] \arrow[lluuu, "z_2"']
    \end{tikzcd} 
  \end{equation*}
\end{defn}
We usually write \begin{tikzcd}A \arrow[r, "i_1"] & A+B & B \arrow[l, "i_2"']\end{tikzcd} for the coproduct and $[f, g]$ for the uniquely determined arrow $u:A+B \to Z.$ The ``coprojections'' $i_1:A \to A+B$ and $i_2:B\to A+B$ are usually called \emph{injections}, with no deeper meaning.

A coproduct is therefore, precisely the product of the objects in the opposite category. This immediately gets a lot of examples of coproducts. However, the opposite category of a familiar category is not really very familiar. Let us look at some more familiar categories and coproducts there.

\hrulefill

However, before we see examples, a joke:
\begin{joke}
A mathematician is a person who turns coffee into theorems.\\
A comathematician is a coperson who turns cotheorems into ffee.
\end{joke}
\hrulefill

\example{}  In $\mathsf{Set},$ the coproduct $A+B$ of two sets is their disjoint union which can be constructed, for example, as
\begin{equation*} 
  A + B = \{(a, 1) \mid a \in A\} \cup \{(b, 2) \mid b \in B\}
\end{equation*}
with evident coproduct injections as
\begin{equation*} 
  i_1(a) = (a, 1), \quad i_2(b) = (b, 2).
\end{equation*}
Given any functions $f$ and $g$ as in
\begin{equation*} 
  \begin{tikzcd}
    A \arrow[rr, "f"] && Z && B \arrow[ll, "g"']
  \end{tikzcd},
\end{equation*}
we define $[f, g]:A+B \to Z$ as
\begin{equation*} 
  [f, g](x, \delta) = \begin{cases}
    f(x) & \delta = 1\\
    g(x) & \delta = 2.
  \end{cases}
\end{equation*}
It can be verified that $[f, g]\circ i_1 = f$ and $[f, g]\circ i_2 = g.$\\
Moreover, given any $h:A+B \to Z$ with $h\circ i_1 = f$ and $h\circ i_2 = g,$ we must have $h = [f, g].$\\
Thus, every pair of objects in $\mathsf{Set}$ does have a coproduct.\\
Also, note that in $\mathsf{Set},$ every finite set is a coproduct:
\begin{equation*} 
  A \cong \underbrace{1 + 1 + \cdots + 1}_{n \text{ times}}
\end{equation*}
for $n = \operatorname{card}(A).$ This is because a function $f:A \to Z$ is uniquely determined by its values $f(a)$ for all $a \in A.$ (This also encapsulates the fact that one may define $f(a)$ in \emph{any} way for each $a \in A$ and still get a function $f:A\to Z.$ This is in contrast to something more structured like a monoid where the arrows must satisfy some further constraints.)

\example{} If $M(A)$ and $M(B)$ are \emph{free} monoids on sets $A$ and $B,$ then in $\mathbf{Mon},$ we can construct there coproduct as
\begin{equation*} 
  M(A) + M(B) \cong M(A+B),
\end{equation*}
where $A+B$ is the coproduct of sets, id est, their disjoint union as defined above.\\
The injections are the natural inclusions.\\
One can see that this is a coproduct directly by considering words over $A + B,$ but it also follows abstractly by using the diagram.
\begin{equation} \label{diag:moncoprod}
  \begin{tikzcd}
    && N&&\\&&&&\\&&&&\\&&&&\\
    M(A) \arrow[rruuuu] \arrow[rr]&  & 
    M(A+B) \arrow[uuuu, dotted]    &  & M(B) \arrow[lluuuu] \arrow[ll]
    \\&&&&\\&&&&\\
    A \arrow[uuu, "\eta_A"] \arrow[rr] &  & A+B \arrow[uuu, "\eta_{A+B}"] 
    && B \arrow[ll] \arrow[uuu, "\eta_B"']
  \end{tikzcd}
\end{equation}
in which the $\eta$s are the respective insertion of generators. (Recall this from \S\S\ref{ssec:free}.)\\
(Note that there's actually an abuse of notation in the above diagram as we objects from both $\mathsf{Set}$ and $\mathsf{Mon}$ in it. This will carry on for the rest of this example.)\\
The UMPs of $M(A), M(B), A+B,$ and $M(A+B)$ then imply that $M(A+B)$ has the required UMP of $M(A) + M(B).$ We look at this in more detail as follows:\\
The injections $i_1 : M(A) \to M(A + B)$ and $i_2 : M(B) \to M(A + B)$ are defined to be precisely those that make the squares in (\ref{diag:moncoprod}) commute. (Their existence and uniqueness are given by the UMP of free monoids.)\\
Now we show that $M(A + B)$ has the desired UMP given these injections.\\
Let $f:M(A) \to N$ and $g:M(B) \to N$ be monoid homomorphisms. We want to show the existence of a unique monoid homomorphism $u:M(A + B) \to N$ that makes the two triangles commute.\\
\textbf{Existence}: Consider the arrows $f\circ\eta_A:A\to N$ and $g\circ\eta_B:B\to N$ (in $\mathsf{Set}$). By the UMP of $A + B,$ there exists $h:A+B \to N$ making the following diagram commute.
\begin{equation} \label{diag:coprodmon2}
  \begin{tikzcd}
  && N &&\\&&&&\\&&&&\\
  M(A) \arrow[rruuu, "f"]            &  &                         &  & M(B) \arrow[lluuu, "g"']\\&&&&\\&&&&\\
  A \arrow[uuu, "\eta_A"] \arrow[rr, "i_1'"'] &  & A+B \arrow[uuuuuu, "h"] &  & B \arrow[ll, "i_2'"] \arrow[uuu, "\eta_B"']
  \end{tikzcd}
\end{equation}
Now, using the UMP of the free monoid $M(A + B),$ get a monoid homomorphism $u:M(A + B) \to N$ such that $u\circ\eta_{A+B} = h.$\\
Now, we show that this $u$ makes the triangles commute. We show this for the left triangle. We first observe that $f\circ \eta_A = u \circ i_1 \circ \eta_A.$ This was because $h = u \circ \eta_{A+B}$ and the fact that the left square commuted. And also note that
\begin{align*} 
  f\circ \eta_A = u \circ i_1 \circ \eta_A \implies f = u \circ i_1.
\end{align*}
This above follows from the UMP of $M(A).$\\
Similarly, we get that the right triangle commutes.\\
\textbf{Uniqueness}: Let $v:M(A+B) \to N$ be another monoid homomorphism making (\ref{diag:moncoprod}) commute. Then (\ref{diag:coprodmon2}) also commuted with $v\circ \eta_{A+B}$ instead of $h.$ However, by the UMP of $M(A+B),$ this forces $h = v\circ \eta_{A+B},$ id est, $u \circ \eta_{A+B} = v \circ \eta_{A+B}.$ This clearly forces $u = v,$ as desired.
  
Note: twice in the above have we used that $f\circ \eta = g\circ\eta \implies f = g.$ This had not been proven earlier but is an easy consequence of the UMP. This is left to the reader.

The foregoing examples says precisely that the free monoid functor $M:\mathsf{Set} \to \mathsf{Mon}$ preserves coproducts.

\example{} In $\mathsf{Top},$ the coproduct of two topological spaces $X$ and $Y$ is the space $X + Y$ defined as follows:\\
As a set, $X + Y$ is simply the disjoint union of $X$ and $Y,$ id est, the coproduct in $\mathsf{Set}.$\\
A set $U \subset X + Y$ is open iff $U \cap X$ is open and $U \cap Y$ is open. (Considering our previous construction of coproduct in $\mathsf{Set},$ we should write $U \cap (X \times \{1\})$ with the understanding that $X \times \{1\}$ has the topology $O(X) \times \{1\}.$)\\
The injections are the same as in $\mathsf{Set}.$ It is an easy verification that these injections are indeed arrows in $\mathsf{Top},$ id est, these are continuous.\\
Moreover, given any $z_1, z_2, Z$ as in the definition, it can be verified that the arrow $u:X+Y \to Z$ obtained in $\mathsf{Set}$ is indeed an arrow in $\mathsf{Top},$ id est, it is continuous.

\example{} Coproducts of posets are similarly constructed from the coproducts of the underlying sets, by ``putting them side by side.'' That is, given posets $P$ and $Q,$ the poset $P+Q$ is simply a poset on the disjoint union $P + Q$ with the relation as inherited from earlier without any additional ones.\\
What about ``rooted posets'', id est, posets with a distinguished initial element $0?$ In the category $\mathsf{Pos}_0$ of such posets and monotone maps that preserve $0,$ one constructs the coproduct of two such posets $P$ and $Q$ from the coproduct $P + Q$ in the category $\mathsf{Pos},$ by ``identifying'' the two different $0$s,

\begin{equation*} 
  A +_{\mathsf{Pos}_0} B = (A +_{\mathsf{Pos}} B)/\text{``}0_A = 0_B\text{''}.
\end{equation*}
(Recall \nameref{equivrel}.)

Recall the example of product in a poset (viewed as a category). There we had gotten the product to be the greatest lowest bound of two elements. Dually, one can consider the question of coproduct in a poset. The answer is not surprising.

\example{} Let $P$ be a fixed poset and $p, q \in P.$ Suppose $p + q$ exists. Then we have

\begin{equation*} 
  p \le p + q \quad \text{and} \quad q \le p + 1
\end{equation*}
and if 

\begin{equation*} 
  p \le z \quad \text{and} \quad p \le z
\end{equation*}
then

\begin{equation*} 
  p + q \le z.
\end{equation*}
So, $p + q = p \vee q$ is the join, or \emph{least upper bound} of $p$ and $q.$\\
(Of course, it is not necessary that joins exist.)
\sec{Acknowledgments}
Here's a list of people who have pointed out typos in the notes so far. I'm thankful to them.
\begin{enumerate}
	\item Ishan Kapnadak
\end{enumerate}
\end{document}