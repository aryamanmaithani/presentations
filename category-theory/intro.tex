\sec{Introduction}
This is essentially a way to force myself to be productive and self-study Category Theory. I try to write these notes in a way that I'm teaching them to someone to help me understand it better.\\
That being said, I don't really think that these notes would be better than reading the book which I'm reading itself. The book being - 
\begin{center}
	\emph{Category Theory} by \emph{Steve Awodey}.
\end{center}
I do skip some of the more advanced examples but in places I do add extra explanations as I feel necessary. So yeah, have fun.\\
A personal suggestion: whenever you see a proposition or lemma or anything that is followed by a proof, try doing it yourself. When you struggle with it, that's when the definitions \emph{really} seep in. Due to this, you would find many proofs that are not exactly the same as the ones given in the book. \\~\\
These notes will keep getting updated as I read more, possibly over the next two years.\\
I'm also not proof-reading these, so there are bound to be many errors. If you find any, \emph{please} let me know. You shall then get to feature in the last section - \nameref{sec:ack}, if you wish. Apart from typos, you may also give suggestions if you think that something was ambiguously phrased. This will help me in wording things better for more clarity. That's how these notes could \emph{really} end up adding something of value.\\
Lastly, if you're on a desktop/laptop and want to view this in your browser instead (so that you can select text or click hyperlinks), you may use the following link - 
\begin{center}
	\url{bit.ly/cat-raw}
\end{center}