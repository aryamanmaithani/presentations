\sec{Some Structures}
\subsection{Monoids}
A \emph{monoid} is a set $M$ equipped with a binary operation $*$ and a distinguished element $u \in M$ satisfying:
\begin{itemize}
	\item $a*(b*c) = (a*b)*c,$ for all $a, b, c \in M,$ and
	\item $a*u = a = u*a$ for all $a \in M.$
\end{itemize}
We may sometimes also refer to the monoid as $(M, *, u).$\\
A monoid homomorphism $h:(M, *, u_M) \to (N, \cdot, u_N)$ is a function $h:M\to N$ satisfying
\begin{itemize}
	\item $h(a*b) = h(a)\cdot h(b),$ for all $a, b \in M,$ and
	\item $h(u_M) = h(u_N).$
\end{itemize}
\subsection{Groups}
With the same notation as earlier, a monoid $M$ is said to be \emph{group} if for every $a \in M,$ there exists some $b \in M$ such that $a * b = u = b * a.$\\
A group homomorphism is monoid homomorphism between groups.
\subsection{Preorders}
A \emph{preorder} is a set $P$ equipped with a binary relation $\le$ satisfying:
\begin{itemize}
	\item $a \le a$ for all $a \in P,$ and
	\item $a \le b$ and $b \le c$ implies $a \le c$ for all $a, b, c \in P.$
\end{itemize}
That is, the relation is reflexive and transitive.
\subsection{Posets}
A \emph{poset} is a preorder $P$ with the additional condition that $\le$ is antisymmetric, that is,
	\begin{equation*} 
		a \le b \text{ and } b \le a \implies a = b \text{ for all } a, b \in P.
	\end{equation*}
An order-preserving map between posets $(P, \le_P)$ and $(Q, \le_Q)$ is a function $f:P\to Q$ satisfying
\begin{align*} 
	p \le_P q \implies f(p) \le_Q f(q). 
\end{align*}
\subsection{Boolean algebra} \label{boolalg}
A \emph{Boolean algebra} is a poset $B$ equipped with distinguished elements $0, 1,$ binary operations $a\vee b$ of ``join'' and $a\wedge b$ of ``meet,'' and a unary operation $\neg b$ of ``complementation.'' These are required to satisfy the conditions
\begin{align*} 
	0 &\le a\\
	a &\le 1\\
	a \le c \text{ and } b \le c \quad &\text{iff} \quad a \vee b \le c \\
	c \le a \text{ and } c \le b \quad &\text{iff} \quad c \le a \wedge b \\
	a \le \neg b \quad &\text{iff} \quad a \wedge b = 0\\
	\neg\neg a  &= a.
\end{align*}
A typical example of a Boolean algebra is the power-set $\mathcal{P}(X)$ of $X.$ We have the following identifications:
\begin{equation*} 
	0 = \emptyset, 1 = X, A \vee B = A \cup B, A \wedge B = A \cap B, \neg A = X\setminus A.
\end{equation*}
A Boolean homomorphism is a function $h:B\to B'$ between Boolean algebras that is an order-preserving map which preserves the addition structure, id est, $h(0) = 0, h(1) = 1, h(a \vee b) = h(a) \vee h(b), h(a \wedge b) = h(a) \wedge h(b), h(\neg a) = \neg h(a).$