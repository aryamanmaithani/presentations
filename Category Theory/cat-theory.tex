\documentclass[11pt,leqno,landscape,semhelv]{seminar}
\usepackage{amsmath, amssymb, amsfonts, amsthm, mathtools}
\usepackage{fancybox}
\usepackage[inline]{enumitem}
\usepackage{cancel}
\usepackage{soul}
\usepackage{centernot}
\usepackage{tikz-cd}

\numberwithin{equation}{section}

\theoremstyle{definition}
\newtheorem{thm}{Theorem}
\newtheorem{lem}[thm]{Lemma}
\newtheorem{cor}[thm]{Corollary}
\newtheorem{prop}[thm]{Proposition}
\newtheorem{defn}[thm]{Definition}

\usepackage{chngcntr}
\numberwithin{thm}{section}
%\numberwithin{defn}{section}
\numberwithin{equation}{section}

\newcommand{\example}[1]{\refstepcounter{thm}\par\medskip
   {\textbf{Example \thethm.} #1} \rmfamily}
\usepackage{titlesec}
\titleformat{\section}[block]
  {}{\centering\huge\S\thesection}{0.25cm}{\centering\Huge\textsc}
\titleformat{\subsection}[block]
  {}{\S\S\thesubsection}{0.25cm}{\Large}
  
\renewcommand{\sec}[1]{%
\begin{slide}
\begin{center}
    \begin{center}
        \section{#1}
    \end{center}
\end{center}
\end{slide}}
\newcommand{\cd}[1]{
\begin{center}
	\begin{tikzcd}
		#1
	\end{tikzcd}
\end{center}}
\newcommand{\cod}{\operatorname{cod}}
\newcommand{\dom}{\operatorname{dom}}
\newcommand{\id}{\operatorname{id}}
\newcommand{\Hom}{\operatorname{Hom}}
\newcommand{\op}{^{\operatorname{op}}}
\newcommand{\mono}{\rightarrowtail}
\newcommand{\epi}{\twoheadrightarrow}

\setlength\parindent{0pt}
\let\emptyset\varnothing

\usepackage{xcolor}
\definecolor{mybgcolor}{RGB}{50, 50, 50} %46, 51, 63
\definecolor{mylinkcolor}{RGB}{0, 255, 255} %46, 51, 63

\usepackage{pagecolor}
%\pagecolor{mybgcolor}
%\color{white}

\usepackage[colorlinks=true,
	%linkcolor=mylinkcolor
	]
	{hyperref}


\title{\vspace{1cm} Category Theory}
\author{Aryaman Maithani\\\url{https://aryamanmaithani.github.io/}}
\date{\today}

\begin{document}
\maketitle
\newpage
\begin{slide}
	\tableofcontents
\end{slide}
\input{sec1.tex}
%
%
%
\sec{Abstract Structures}
\subsection{Epis and Monos}
\begin{defn} 
	In any category $C,$ an arrow
	\begin{equation*} 
		f:A\to B
	\end{equation*}
	is called a 
	\begin{itemize}
		\item \emph{monomorphism}, if given any $g, h:C\to A,$ $fg = fh$ implies $g = h.$
		\item \emph{epimorphism}, if given any $i, j:B\to D,$ $if = jf$ implies $i = j.$
	\end{itemize}
	\begin{equation*} 
		\begin{tikzcd}
		C \arrow[rr, "h"', shift right] \arrow[rr, "g", shift left] &  & A \arrow[rr, "f"] &  & B \arrow[rr, "i", shift left] \arrow[rr, "j"', shift right] &  & D
		\end{tikzcd}
	\end{equation*}
\end{defn}	
We often write $f:A\mono B$ if $f$ is a monomorphism and $f:A \epi B$ if $f$ is an epimorphism.
\begin{prop} \label{prop:monin}
	A function $f:A\to B$ between is monic iff $f$ is injective.
\end{prop}
\begin{proof} 
	$(\implies)$ Suppose that $f$ is monic. We show that $f$ is injective. \\
	Let $a, a' \in A$ be such that $f(a) = f(a').$\\
	Consider $g:A \to A$ defined as
	\begin{align*} 
		g(x) = \begin{cases}
			x & x \neq a\\
			a' & x = a
		\end{cases}
	\end{align*}
	and let $h:A \to A = 1_A.$\\
	Clearly, one sees that $fg = f = f1_A.$ As $f$ is monic, we have that $1_A = g.$ In particular, $g(a) = 1_A(a)$ which yields $a' = a.$\\
	$(\impliedby)$ Suppose that $f$ in injective. We show that $f$ is monic.\\
	Let $g, h:C\to A$ be arrows such that $fg = fh.$ Let $a \in A$ be arbitrary. Then, $fg(a) = fh(a).$ As $f$ is injective, this yields $g(a) = h(a).$
\end{proof}
Before going ahead, we may make the following observation for proving that $f:A\to B$ is injective.
\begin{prop} \label{prop:injec}
	Let $f:A\to B$ be a function. Let $1 = \{*\}$ be any one-element set.\\
	Suppose that $fg \neq fh$ whenever $g, h: 1 \to A$ are distinct functions.\\
	Then, $f$ is injective.
\end{prop}
\begin{proof} 
	Let $a, a' \in A$ be such that $a \neq a'.$ Consider the functions
	\begin{equation*} 
		\bar{a}, \bar{a'} : 1 \to A
	\end{equation*}
	where
	\begin{equation*} 
		\bar{a}(*) = a, \quad \bar{a'}(x) = a'.
	\end{equation*}
	Since $\bar{a} \neq \bar{a'},$ it follows from our hypothesis that $f\bar{a} \neq f\bar{a'}.$ Thus, $f(a) = (f\bar{a})(x) \neq (f\bar{a'})(x) = f(a').$ Hence, it follows that $f$ is injective.
\end{proof}
Using this proposition, one may give a slightly simpler proof of $(\implies)$ of Proposition \ref{prop:monin}.\\
\example{} In many categories of ``structured sets'' like monoids, the monos are exactly the ``injective homomorphisms''. More precisely, a homomorphism $h:M\to N$ is monic precisely if the underlying function $|h|:M\to N$ is injective. (By the above, it is the same as saying $|h|$ is monic.)\\
To see that $|h|$ being injective $\implies$ $h$ is monic, one may consider the earlier proof.\\
Conversely, let $h:M \to N$ be monic. We show that $|h|$ is monic. \\
Suppose $x, y:1 \to |M|$ are two different ``elements'' (arrows), where $1 = \{*\},$ any one-element set. By Proposition \ref{prop:injec}, it suffices to prove that $|h|x, |h|y : 1 \to N$ are also distinct.\\
By the UMP of the free monoid $M(1),$ there are distinct corresponding homomorphisms $\bar{x}, \bar{y}:M(1) \to M,$ with distinct composites $h\circ \bar{x}, h\circ \bar{y}:M(1) \to N,$ since $h$ is monic. Thus, the corresponding ``elements'' $|h|\circ x, |h|\circ y:1 \to N$ are also distinct, again by the UMP of $M(1).$
\end{document}